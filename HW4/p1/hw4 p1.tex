\documentclass[11pt]{article}
\usepackage{amsmath, amsfonts, amsthm, amssymb}  % Some math symbols
\usepackage[utf8x]{inputenc}
\usepackage{fullpage}
\usepackage[x11names, rgb]{xcolor}
\usepackage{graphicx}
\usepackage{tikz}
\usepackage{etoolbox}
\usepackage{enumerate}
\usepackage{enumitem}
\usepackage{listings}
\usepackage{hyperref}
\usepackage{lipsum}
\usepackage{sectsty}
\usepackage{verbatim}
\usetikzlibrary{decorations,arrows,shapes}

%% Define the title contents
\title{}
\author{}
\date{}

%% Set the size of the section header
\sectionfont{\fontsize{11}{12}\selectfont}

%% Set the size and format of the subsection header
\subsectionfont{\fontsize{11}{12}\selectfont}
\renewcommand{\thesubsection}{\thesection (\alph{subsection})}

%% Set the size and format of the subsubsection header
\subsubsectionfont{\fontsize{9}{10}\selectfont}
\renewcommand{\thesubsubsection}{\roman{subsubsection}}

%% Define Real and Rational numbers symbol
\newcommand{\R}{\mathbb{R}}
\newcommand{\Q}{\mathbb{Q}}
\newcommand{\N}{\mathbb{N}}
\newcommand{\Z}{\mathbb{Z}}

%% Redefine rightarrow to imp
\def\imp{\rightarrow}

%% Redefine overline to ol
\def\ol{\overline}

%% Redefine leftrightarrow to lra
\def\lra{\leftrightarrow}

% Redefine setminus to sm
\def\sm{\setminus}

%% Define a nested environment using level for formal proof
\newenvironment{level}%
{\addtolength{\itemindent}{2em}}%
{\addtolength{\itemindent}{-2em}}



%% Set enumerate sub list to use numbers instead of letters
\setlist[enumerate]{label*=\arabic*.}

%% Define custom style
\lstdefinestyle{myCustomMatlabStyle}{
  language=Java,
  numbersep=10pt,
  tabsize=4,
  showspaces=false,
  showstringspaces=false
}

%% Define the default code language to Java
\lstset{basicstyle=\small, style=myCustomMatlabStyle}

%%--- Begin the Document ---%%

\begin{document}

\section*{P1}

\noindent\textcolor[RGB]{220,220,220}{\rule{\linewidth}{0.8pt}}

\subsection*{Claim:} 

For any pair of $x, y \in \R$ such that $0 \leq x < y$, $x^2 < y^2$.

\subsection*{Proof:}

Assume for a contradiction, that there exists a pair of $x, y \in \R$ such that $0 \leq x < y$ and $x^2 \geq y^2$. Let $x$ be less than $y$ by some positive, nonzero amount $k$ such that $x + k = y$. Then, by substitution $x^2 \geq (x + k)^2$. Algebra shows this to be equivalent to $-\frac{k}{2} \geq x$. Since $k$ is positive, the term $-\frac{k}{2}$ is always negative, but then $x$ is negative which is a contradiction. $\qed$

\noindent\textcolor[RGB]{220,220,220}{\rule{\linewidth}{0.8pt}}

\subsection*{Claim:} 

The minimum spanning tree produced by Kruskal's algorithm on graph $G$ where the cost of edge $e$ is $c_e$, and all $c_e$ are positive and distinct, will be the same mininimum spanning tree produced by Kruskal's algorithm on $G$ after updating the cost of every $e$ to $c_e^2$.

\subsection*{Proof:}

Let Kruskal's algorithm produce a minimum spanning tree $T$ from arbitrary connected graph $G = (V, E)$. Let $E'$ refer to the set of edges under consideration for inclusion in $T$ during an arbitrary iteration of the algorithm. Let edge $e$ be the minimum cost edge in $E'$ such that the cost $c_e < c_f$ for any other edge $f \in E'$ and, by the cut property, $e$ is in $T$. If the cost of every edge in $E$ is squared, such that in $E'$ $c_e$ becomes $c_e^2$ and $c_f$ becomes $c_f^2$, then by the above proof $c_e^2 < c_f^2$ and $e$ is still the minimum cost edge in $E'$. Thus, the minimum spanning tree produced by Kruskal's algorithm is the same. $\qed$
\end{document}