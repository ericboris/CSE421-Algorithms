\documentclass[11pt]{article}
\usepackage{amsmath, amsfonts, amsthm, amssymb}  % Some math symbols
\usepackage[utf8x]{inputenc}
\usepackage{fullpage}
\usepackage[x11names, rgb]{xcolor}
\usepackage{graphicx}
\usepackage{tikz}
\usepackage{etoolbox}
\usepackage{enumerate}
\usepackage{enumitem}
\usepackage{listings}
\usepackage{hyperref}
\usepackage{lipsum}
\usepackage{sectsty}
\usepackage{verbatim}
\usepackage{mathtools}
\usetikzlibrary{decorations,arrows,shapes}

%% Define the title contents
\title{}
\author{}
\date{}

%% Set the size of the section header
\sectionfont{\fontsize{11}{12}\selectfont}

%% Set the size and format of the subsection header
\subsectionfont{\fontsize{11}{12}\selectfont}
\renewcommand{\thesubsection}{\thesection (\alph{subsection})}

%% Set the size and format of the subsubsection header
\subsubsectionfont{\fontsize{9}{10}\selectfont}
\renewcommand{\thesubsubsection}{\roman{subsubsection}}

%% Define Real and Rational numbers symbol
\newcommand{\R}{\mathbb{R}}
\newcommand{\Q}{\mathbb{Q}}
\newcommand{\N}{\mathbb{N}}
\newcommand{\Z}{\mathbb{Z}}

%% Define ceiling and floor delimiters
\DeclarePairedDelimiter\ceil{\lceil}{\rceil}
\DeclarePairedDelimiter\floor{\lfloor}{\rfloor}

%% Redefine rightarrow to imp
\def\imp{\rightarrow}

%% Redefine overline to ol
\def\ol{\overline}

%% Redefine leftrightarrow to lra
\def\lra{\leftrightarrow}

% Redefine setminus to sm
\def\sm{\setminus}

%% Define a nested environment using level for formal proof
\newenvironment{level}%
{\addtolength{\itemindent}{2em}}%
{\addtolength{\itemindent}{-2em}}



%% Set enumerate sub list to use numbers instead of letters
\setlist[enumerate]{label*=\arabic*.}

%% Define custom style
\lstdefinestyle{myCustomMatlabStyle}{
  language=Java,
  numbersep=10pt,
  tabsize=4,
  showspaces=false,
  showstringspaces=false
}

%% Define the default code language to Java
\lstset{basicstyle=\small, style=myCustomMatlabStyle}

%%--- Begin the Document ---%%

\begin{document}

\section*{P4}

\noindent\textcolor[RGB]{220,220,220}{\rule{\linewidth}{0.8pt}}

\subsection*{Prompt:} 
Given an array $a_1$, ..., $a_n$ of n distinct integers, design an O(log $n$) time algorithm that finds $a_i$ such that $a_i > a_{i-1}$ and $a_i > a_{i+1}$.

\subsection*{Algorithm:} 
\begin{lstlisting}[basicstyle=\small, mathescape=true]
// Given array A and range indices lo and hi, find a local max $a_i$
LocalMax(A, $lo$, $hi$):
	// Handle the base cases
	If $hi - lo < 0$ then
		Return "Impossible"
	Else if $hi - lo = 0$ then
		Return A[$lo$]
	EndIf

	// Check the upper and lower bounds of A[lo:hi]
	If A[$lo$] > A[$lo + 1$] then
		Return A[$lo$]
	Else if A[$hi$] > A[$hi - 1$] then
		Return A[$hi$]
	EndIf

	// Check the middle index of A and recurse
	// Let mid always be an integer
    Let $mid = lo + (hi - lo)$ / $2$
	If A[$mid$] > A[$mid - 1$] and A[$mid$] > A[$mid + 1$] then
		Return A[$mid$]
	Else if A[$mid - 1$] > A[$mid + 1$] then
        Let $hi = mid - 1$
		Return LocalMax(A, $lo$, $hi$)
	Else then
        Let $lo = mid + 1$
		Return LocalMax(A, $lo$, $hi$)
	EndIf
\end{lstlisting}

\noindent\textcolor[RGB]{220,220,220}{\rule{\linewidth}{0.8pt}}
\linebreak

\subsection*{Claim:} 

The algorithm terminates in O(log $n$) time.

\subsection*{Proof:} 

The algorithm reduces the problem into a variant of binary search which is known to terminate in O(log $n$). That is, it solves a problem of size $n$ by reducing it into one subproblem of half the size, which it recursively solves, and combines each level in constant time. We can express the runtime of the algorithm as a recurrence relation of the form $T(n) \leq T(\ceil*{\frac{n}{2}}) + O(1)$ when $n > 1$, and $T(1) \leq c$. Which, by the master theorem is O(log $n$). $\qed$

\noindent\textcolor[RGB]{220,220,220}{\rule{\linewidth}{0.8pt}}

\subsection*{Claim:} 

Given an array $a_1$, ..., $a_n$ of n distinct integers, the algorithm finds an $a_i$ such that $a_i > a_{i-1}$ and $a_i > a_{i+1}$. If $a_1 > a_2$ or $a_n > a_{n-1}$ the algorithm returns $a_1$ or $a_n$, respectively.

\subsection*{Proof:} 
Let P(n) be the claim that, given an array $A = a_1$, ..., $a_n$ of $n$ distinct integers, if $a_1 > a_2$ or $a_n > a_{n-1}$ the algorithm returns $a_1$ or $a_n$, respectively, otherwise, the algorithm returns an $a_i$ such that $a_i > a_{i-1}$ and $a_i > a_{i+1}$. Prove P(n) by induction.
\newline\newline
Base Case: P(1) holds because $a_1 = a_i = a_n$ is the maximum value by dint of being the only value. 
\newline\newline
Inductive Hypothesis: For some $k \geq 1$, assume P($j$) holds for $1 \leq j \leq k-1$.
\newline\newline
Inductive Step: Goal, show P($k$) holds. 
Let indices $hi$ and $lo$ represent the upper and lower bounds of $A$ such that $k = hi - lo$. If, by simple comparison, $a_{lo} > a_{lo + 1}$ or $a_{hi} > a_{hi - 1}$ a local maximum, $a_{lo}$ or $a_{hi}$, is immediately found and we are done. If not, let index $mid = lo - \frac{hi - lo}{2}$. If $a_{mid} > a_{mid - 1}$ and $a_{mid} > a_{mid + 1}$ then $a_{mid}$ is a local maximum and we are done. Otherwise, we check either the range to the left or right of $mid$. If $mid - 1 > mid + 1$, check the range between $lo$ and $mid - 1$ for a maximum. Because $(mid - 1) - lo = (lo + \frac{hi - lo}{2} - 1) - lo = \frac{k}{2} - 1 < k$, P($\frac{k}{2} - 1$) holds. Or if $mid - 1 \leq mid + 1$, check the range between $mid + 1$ and $hi$ for a maximum. Because $hi - (mid + 1) = hi - (lo + \frac{hi - lo}{2}) - 1 = \frac{k}{2} - 1 < k$, P($\frac{k}{2} - 1$) holds. $\qed$
\end{document}