\documentclass[11pt]{article}
\usepackage{amsmath, amsfonts, amsthm, amssymb}  % Some math symbols
\usepackage[utf8x]{inputenc}
\usepackage{fullpage}
\usepackage[x11names, rgb]{xcolor}
\usepackage{graphicx}
\usepackage{tikz}
\usepackage{etoolbox}
\usepackage{enumerate}
\usepackage{enumitem}
\usepackage{listings}
\usepackage{hyperref}
\usepackage{lipsum}
\usepackage{sectsty}
\usepackage{verbatim}
\usetikzlibrary{decorations,arrows,shapes}

%% Define the title contents
\title{}
\author{}
\date{}

%% Set the size of the section header
\sectionfont{\fontsize{11}{12}\selectfont}

%% Set the size and format of the subsection header
\subsectionfont{\fontsize{11}{12}\selectfont}
\renewcommand{\thesubsection}{\thesection (\alph{subsection})}

%% Set the size and format of the subsubsection header
\subsubsectionfont{\fontsize{9}{10}\selectfont}
\renewcommand{\thesubsubsection}{\roman{subsubsection}}

%% Define Real and Rational numbers symbol
\newcommand{\R}{\mathbb{R}}
\newcommand{\Q}{\mathbb{Q}}
\newcommand{\N}{\mathbb{N}}
\newcommand{\Z}{\mathbb{Z}}

%% Redefine rightarrow to imp
\def\imp{\rightarrow}

%% Redefine overline to ol
\def\ol{\overline}

%% Redefine leftrightarrow to lra
\def\lra{\leftrightarrow}

% Redefine setminus to sm
\def\sm{\setminus}

%% Define a nested environment using level for formal proof
\newenvironment{level}%
{\addtolength{\itemindent}{2em}}%
{\addtolength{\itemindent}{-2em}}



%% Set enumerate sub list to use numbers instead of letters
\setlist[enumerate]{label*=\arabic*.}

%% Define custom style
\lstdefinestyle{myCustomMatlabStyle}{
  language=Java,
  numbersep=10pt,
  tabsize=4,
  showspaces=false,
  showstringspaces=false
}

%% Define the default code language to Java
\lstset{basicstyle=\small, style=myCustomMatlabStyle}

%% Remove indentation at start of paragraph
\usepackage[parfill]{parskip}

%%--- Begin the Document ---%%

\begin{document}

\section*{P3}

\noindent\textcolor[RGB]{220,220,220}{\rule{\linewidth}{0.8pt}}

\subsection*{Claim:} 

4-Color $\leq_\text{p}$ 5-Color.

\subsection*{Proof:}

\begin{lstlisting}[basicstyle=\small, mathescape=true]
Suppose G = (V, E) is a 4-colorable graph
If G is 4-Colorable, then G' is 5-Colorable
	since G is 4-colorable, there exists a color mapping c such that
		c: V = {1, 2, 3, 4} where 
		for all vertices v in V
			for all vertices u in V
				if v != u then
					c(v) != c(u)
	Let graph G' = (V', E') such that
		let vertex x be a vertex not in V
		V' = V $\cup$ x
		E' = E $\cup$ {(e, x) for all e in E}
	Let there be a mapping c' such that
		c': V' = {1, 2, 3, 4, 5} where
		for all vertices v in V'
			for all vertices u in V'
				if v != u then
					c(v) != c(u)
		and such that c(v) = c'(v) for all v in V and
		c'(x) = 5 for the vertex x in V'
	By construction c' is a 5-coloring of G'
	Thus if G is 4-Colorable, then G' is 5-colorable

Suppose graph G' = (V', E') is a 5-colorable graph
If G' is 5-colorable, then G is 4-colorable
	since G' is 5' colorable, there exists a color mapping c' such that
		c': V = {1, 2, 3, 4, 5} where 
		for all vertices v in V
			for all vertices u in V
				if v != u then
					c(v) != c(u)
		and such that without loss of generality c'(x) = 5
	x shares an edge with every other vertex in V'
	so no other vertex v in V' has c'(v) = 5
	Let there be a mapping c such that
		c: V = {1, 2, 3, 4} where 
		for all vertices v in V
			for all vertices u in V
				if v != u then
					c(v) != c(u)
		and such that for all vertices v in V, c(v) = c'(v)
	c is a 4 coloring for G
	Thus, if G' is 5 colorable, then G is 4-colorable
\end{lstlisting}

\end{document}