\documentclass[11pt]{article}
\usepackage{amsmath, amsfonts, amsthm, amssymb}  % Some math symbols
\usepackage[utf8x]{inputenc}
\usepackage{fullpage}
\usepackage[x11names, rgb]{xcolor}
\usepackage{graphicx}
\usepackage{tikz}
\usepackage{etoolbox}
\usepackage{enumerate}
\usepackage{enumitem}
\usepackage{listings}
\usepackage{hyperref}
\usepackage{lipsum}
\usepackage{sectsty}
\usepackage{verbatim}
\usetikzlibrary{decorations,arrows,shapes}

%% Define the title contents
\title{}
\author{}
\date{}

%% Set the size of the section header
\sectionfont{\fontsize{11}{12}\selectfont}

%% Set the size and format of the subsection header
\subsectionfont{\fontsize{11}{12}\selectfont}
\renewcommand{\thesubsection}{\thesection (\alph{subsection})}

%% Set the size and format of the subsubsection header
\subsubsectionfont{\fontsize{9}{10}\selectfont}
\renewcommand{\thesubsubsection}{\roman{subsubsection}}

%% Define Real and Rational numbers symbol
\newcommand{\R}{\mathbb{R}}
\newcommand{\Q}{\mathbb{Q}}
\newcommand{\N}{\mathbb{N}}
\newcommand{\Z}{\mathbb{Z}}

%% Redefine rightarrow to imp
\def\imp{\rightarrow}

%% Redefine overline to ol
\def\ol{\overline}

%% Redefine leftrightarrow to lra
\def\lra{\leftrightarrow}

% Redefine setminus to sm
\def\sm{\setminus}

%% Define a nested environment using level for formal proof
\newenvironment{level}%
{\addtolength{\itemindent}{2em}}%
{\addtolength{\itemindent}{-2em}}



%% Set enumerate sub list to use numbers instead of letters
\setlist[enumerate]{label*=\arabic*.}

%% Define custom style
\lstdefinestyle{myCustomMatlabStyle}{
  language=Java,
  numbersep=10pt,
  tabsize=4,
  showspaces=false,
  showstringspaces=false
}

%% Define the default code language to Java
\lstset{basicstyle=\small, style=myCustomMatlabStyle}

%% Remove indentation at start of paragraph
\usepackage[parfill]{parskip}

%%--- Begin the Document ---%%

\begin{document}

\section*{P2}

\noindent\textcolor[RGB]{220,220,220}{\rule{\linewidth}{0.8pt}}

\subsection*{Prompt:} 

Given a square n x n table, design an algorithm that runs in time polynomial in n and outputs the minimum number of extra squares to remove from the table such that we cannot put any domino in the remaining table.

\subsection*{Algorithm:} 

\begin{lstlisting}[basicstyle=\small, mathescape=true]
Input: A square n x n table T meeting the given description
	   with "X"s on some of the cells
Output: The minimum number of squares that can be removed from T
		such that no domino can be placed on T
MinSquares(T):
	Let X be an empty set
	Let Y be an empty set
	Let E be an empty set
	Color the cells of T in black and white checkerboard pattern
	For each cell c in T do
		If c is black then
			Add a vertex x corresponding to c to X
		Else then
			Add a vertex y corresponding to c to Y
		EndIf
	EndFor
	For each cell c in T do
		If there is no "X" on c then
			For each cell n adjacent to c do
				If there is no "X" on n then
					Let v be the vertex from X or Y corresponding to c
					Let u be the vertex from Y or X corresponding to n
					If there isn't an edge in E between v and u
						Let e be an edge between v and u
						Add e to E
					EndIf
				EndIf
			EndFor
		EndIf
	EndFor
	Let graph G = (X, Y, E)
	Let flow network H = FlowNetwork(G)
	Let min s-t cut C = FFA(H)
	Let min vertex cover S = VertexCover(H, C)
	Output |S|	
\end{lstlisting}

\noindent\textcolor[RGB]{220,220,220}{\rule{\linewidth}{0.8pt}}
\linebreak

\subsection*{Claim:} 

The algorithm terminates in time polynomial in n. 

\subsection*{Proof:} 

We know from HW8 P1 that FlowNetwork, FFA, and VertexCover all terminate in polynomial time. Therefore, that MinSquares terminates is evident given that every for loop iteration over the cells of T is over finite and monotonically decreasing sets. The first outer for loop iterates over each cell for O(n$^2$) runtime while the second outer for loop iterates over each cell times each cell's 4 neighbors for O(n$^2$) runtime. O(n$^2$) is the dominant runtime factor. Thus, MinSquares terminates in time polynomial in n. 

\noindent\textcolor[RGB]{220,220,220}{\rule{\linewidth}{0.8pt}}

\subsection*{Claim:} 

Given a table T, the algorithm outputs the minimum number of squares to remove from T such that no domino can be placed on T. 

\subsection*{Proof:} 

\begin{lstlisting}[basicstyle=\small, mathescape=true]
Let T be colored in black and white checkerboard pattern

The graph G = (X, Y, E) is bipartite
	G is divided into sets X and Y
		every vertex x in X corresponds to a black cell in T and
		every vertex y in Y correspondes to a white cell in T 
		there is no vertex v in both X and Y
		thus G is divided into sets X and Y
	Every edge e is between some x in X and some y in Y
		let v be a vertex in X or Y corresponding to a cell c in T
		let u be a vertex in X or Y corresponding a cell n adjacent to c 
		let e be between v and u
		since no two black cells nor two white cells are adjacent
		there are no edges between any xi and xj in X
		and there are no edges between any yi and yj in Y
		thus every edge in E is between some x in X and some y in Y
	Thus G is bipartite

Every edge in G represents a valid domino placement in T
	an edge is added to G iff there are two available adjacent cells in T
	a valid domino placement requires two adjacent cells in T
	Thus an edge in G represents a valid domino placement in T
	
Let flow network H = FlowNetwork(G)
H is a flow network of G
	We've shown in lecture 23 that H is a flow network of G

Let min s-t cut C = FFA(H)
C is the min s-t cut of H
	We've shown in lecture 23 that Ford-Fulkerson produces min s-t cut

Let min vertex cover S = VertexCover(H, C)
S is the min vertex cover of H
	We've shown in P1 that VertexCover produces the min vertex cover of H

G \ S removes all edges from G
	for any vertex v in G
	removing v from G removes the edges incident to v from G
	S is a vertex cover
	therefore every edge in G has at least one vertex in S
	thus G \ S removes all edges from G

|S| is a number of cells such that no domino can be placed on T
	Every edge in G represents a valid domino placement in T
	G \ S removes all edges from G
	So G \ S represents T with no valid domino placements
	Thus |S| is a number of cells such that no domino can be placed on T
	
Let K be the minimum set of cells that can be removed from T
such that no domino can be placed on T

K corresponds to a vertex cover
	Every cell c in T is represented in X or Y
	Every valid domino placement in T is represented as an edge in E
	every k in K is a cell c in T
	so every k in K
		is represented as a vertex in X or Y and
		represents removing $\geq$ 1 valid domino placement from T
		and so represents removing $\geq$ 1 edge from E
	By definition 
		K is the minimum set of cells to remove all valid domino positions
		so T \ K removes all valid domino positions from T
	So T \ K is equivalent to G \ S
	Thus K corresponds to a vertex cover

|S| = |K|
	Let |min number cells| denote the optimum solution to T

	|min vertex cover| $\leq$ |K|
		suppose S is the minimum vertex cover of H
		then |S| = |min vertex cover|
		Then there are two options
			either |S| = |K|
			or |S| != |K|
		If |S| = |K| then
			|K| = |min vertex cover|
			and since K is the number of cells to remove from T
			|min vertex cover| = |min number cells|
		Else if |S| != |K| then
			since K represent number of cells to remove from T
			it must be that |S| < |K|
				since it can't be that |min vertex cover| > |K|
				so then |min vertex cover| < |K|
		So since |S| $\leq$ |K|
		and since |S| = |min vertex cover|
		Thus |min vertex cover| $\leq$ |K|
	
	|min number cells| $\leq$ |S|
		suppose K is the minimum number of cells to remove from T
		then |K| = |min number cells|
		Then there are two options
			either |K| = |S|
			or |K| != |S|
		If |K| = |S| then
			|S| = |min number cells|
			and since S is the vertex cover of G
			|min number cells| = |min vertex cover|
		Else if |K| != |S| then
			since S represents the vertex cover of G
			it must be that |K| < |S|
				since it can't be that |min number cells| > |S|
				so then |min number cells| < |S|
		So since |K| $\leq$ |S|
		and since |K| = |min number cells|
		Thus |min number cells| $\leq$ |S|
	
	So |S| = |min vertex cover| and
	|K| = |min number cells| and
	|min vertex cover| $\leq$ |K| and 
	|min number cells| $\leq$ |S|
	Then it must be that
		|min vertex cover| = |min number cells| and
		|S| = |K|
	Thus |S| = |K|
\end{lstlisting}

\end{document}
