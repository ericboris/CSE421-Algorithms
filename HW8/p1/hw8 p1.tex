\documentclass[11pt]{article}
\usepackage{amsmath, amsfonts, amsthm, amssymb}  % Some math symbols
\usepackage[utf8x]{inputenc}
\usepackage{fullpage}
\usepackage[x11names, rgb]{xcolor}
\usepackage{graphicx}
\usepackage{tikz}
\usepackage{etoolbox}
\usepackage{enumerate}
\usepackage{enumitem}
\usepackage{listings}
\usepackage{hyperref}
\usepackage{lipsum}
\usepackage{sectsty}
\usepackage{verbatim}
\usetikzlibrary{decorations,arrows,shapes}

%% Define the title contents
\title{}
\author{}
\date{}

%% Set the size of the section header
\sectionfont{\fontsize{11}{12}\selectfont}

%% Set the size and format of the subsection header
\subsectionfont{\fontsize{11}{12}\selectfont}
\renewcommand{\thesubsection}{\thesection (\alph{subsection})}

%% Set the size and format of the subsubsection header
\subsubsectionfont{\fontsize{9}{10}\selectfont}
\renewcommand{\thesubsubsection}{\roman{subsubsection}}

%% Define Real and Rational numbers symbol
\newcommand{\R}{\mathbb{R}}
\newcommand{\Q}{\mathbb{Q}}
\newcommand{\N}{\mathbb{N}}
\newcommand{\Z}{\mathbb{Z}}

%% Redefine rightarrow to imp
\def\imp{\rightarrow}

%% Redefine overline to ol
\def\ol{\overline}

%% Redefine leftrightarrow to lra
\def\lra{\leftrightarrow}

% Redefine setminus to sm
\def\sm{\setminus}

%% Define a nested environment using level for formal proof
\newenvironment{level}%
{\addtolength{\itemindent}{2em}}%
{\addtolength{\itemindent}{-2em}}



%% Set enumerate sub list to use numbers instead of letters
\setlist[enumerate]{label*=\arabic*.}

%% Define custom style
\lstdefinestyle{myCustomMatlabStyle}{
  language=Java,
  numbersep=10pt,
  tabsize=4,
  showspaces=false,
  showstringspaces=false
}

%% Define the default code language to Java
\lstset{basicstyle=\small, style=myCustomMatlabStyle}

%% Remove indentation at start of paragraph
\usepackage[parfill]{parskip}

%%--- Begin the Document ---%%

\begin{document}

\section*{P1}

\noindent\textcolor[RGB]{220,220,220}{\rule{\linewidth}{0.8pt}}

\subsection*{Prompt:} 

Design a polynomial time algorithm to find the minimum vertex cover in a bipartite graph G = (X, Y, E).

\noindent\textcolor[RGB]{220,220,220}{\rule{\linewidth}{0.8pt}}

\subsection*{Algorithm}

Let FFA(H) be the Ford-Fulkerson algorithm which is given a network H = (X, Y, E) with flow capacity c, a source node s, and a sink node t and returns the minimum s-t 
cut (A, B).

\begin{lstlisting}[basicstyle=\small, mathescape=true]
Input: The bipartite graph G = (X, Y, E)
Output: The flow network H
FlowNetwork(G):
	Let H be a copy of G
	For each edge e in H.E
		Let x be the vertex connected to e in H.X
		Let y be the vertex connected to e in H.Y
		Let e be the directed edge (x, y)
		Let e.capacity = $\infty$
	EndFor	
	Add vertex s to H
	For each vertex x in H.X do
		Let e be the directed edge (s, x)
		Let e.capacity = 1
	EndFor
	Add vertex t to H
	For each vertex y in H.Y do
		Let e be the directed edge (y, t)
		Let e.capacity = 1
	EndFor
	Return H
	
Input: The flow network H = (X, Y, E) and the min s-t cut (A, B) in H
Output: A vertex cover S of G such that |S| = cap(A, B)
VertexCover(H, A, B):
	Let X$_\text{B}$ = H.X $\cap$ B
	Let Y$_\text{A}$ = H.Y $\cap$ A
	Let S = X$_\text{B}$ $\cup$ Y$_\text{A}$
	Return S
	
Input: The bipartite graph G = (X, Y, E) and the min vertex cover S of G
Output: The min s-t cut (A, B) in H
MinCut(G, S):
	Let H = FlowNetwork(G)
	Let A = H.s $\cup$ (H.X \ S) $\cup$ (H.Y $\cap$ S)
	Let B = H.t $\cup$ (H.X $\cap$ S) $\cup$ (H.Y \ S)
	Return (A, B)

Let C = FFA(H)
Let S = VertexCover(H, C)
Output S
\end{lstlisting}

\noindent\textcolor[RGB]{220,220,220}{\rule{\linewidth}{0.8pt}}
\linebreak

\subsection*{Claim:} 

The algorithm terminates in polynomial time. 

\subsection*{Proof:} 

That FlowNetwork terminates in polynomial time is easily seen, given that each loop iterates over finite, monotonically decreasing sets. Therefore, FlowNetwork terminates in O($|$X$|$ + $|$Y$|$ + $|$E$|$) which is polynomial. That VertexCover and MinCut terminate in constant time is obvious. We know that Ford-Fulkerson terminates in polynomial time from HW7 P3. Thus, the the algorithm terminates in polynomial time. 

\noindent\textcolor[RGB]{220,220,220}{\rule{\linewidth}{0.8pt}}

\subsection*{Claim:} 

Given a bipartite graph G = (X, Y, E), the algorithm outputs the minimum vertex cover of G. 

\subsection*{Proof:}

\begin{lstlisting}[basicstyle=\small, mathescape=true]
From lecture 23 FlowNetwork algorithm is correct
	Let flow network H = (X, Y, E) be the output of FlowNetwork(G) 
	
From lecture 23 Ford-Fulkerson algorithm is correct
	Let min s-t cut (A, B) be the output of FFA(H)
	cap(A, B) = cap(min cut)	

Let S = (H.X $\cap$ B) $\cup$ (H.Y $\cap$ A).

S is a vertex cover	
	Let X$_A$ = X $\cap$ A
	Let X$_B$ = X $\cap$ B
	Let Y$_A$ = Y $\cap$ A
	Let Y$_B$ = Y $\cap$ B
	There are no edges between X$_A$ and Y$_B$
	For every vertex xa in X$_A$
		For every edge e between X$_A$ and Y$_A$
			Let ya be the vertex connected to e in Y$_A$
			ya is in S
	For every vertex ya in Y$_A$
		For every edge e incident to ya
			ya is in S
	For every vertex xb in X$_B$
		For every edge e incident to xb
			xb is in S
	For every vertex yb in Y$_B$
		For every edge e incident Y$_B$ and X$_B$
			let xb be the vertex connected to e in X$_B$
			xb is in S
	Every edge between X and Y has a vertex in S
	thus S is a vertex cover

|S| = cap(A, B)
	|S| = $|$X$_B|$ + $|$Y$_A|$
	cap(A, B) = $|$X$_B|$ + $|$Y$_A|$
	thus |S| = cap(A, B)

Since |S| = cap(A, B)
	|S| = cap(min cut)

There are two possibilities
	Either |S| = |min vertex cover|
	Or |S| != |min vertex cover|
	
If |S| = |min vertex cover| then
	since |S| = cap(min cut)
	|min vertex cover| = cap(min cut)
Else then
	|min vertex cover| !> |S|
	So |min vertex cover| < |S|
	Then |min vertex cover| < cap(min cut)
		since |S| = cap(min cut)
		|min vertex cover| < cap(min cut)

Thus, given min s-t cut (A, B)
	We find vertex cover S such that
		|S| = cap(A, B) which implies that
		|S| = cap(min cut)
	and
		|min vertex cover| $\leq$ |S| which implies that
		|min vertex cover| $\leq$ cap(min cut)
	
Reset previously used variables S, A, B

Now, suppose |S| = |min vertex cover|

Let (A, B) be an s-t cut such that
	Let A = H.s $\cup$ (H.X \ S) $\cup$ (H.Y $\cap$ S)
	Let B = H.t $\cup$ (H.X $\cap$ S) $\cup$ (H.Y \ S)
		
(A, B) is an s-t cut since
	s is in A
	and t is in B
	thus (A, B) is an s-t cut
		
cap(A, B) = |S|
	cap(A, B) = $|$X$_B|$ + $|$Y$_A|$
	|S| = $|$X$_B|$ + $|$Y$_A|$
	thus cap(A, B) = |S|
		
Since cap(A, B) = |S|
	cap(A, B) = |min vertex cover|
	
There are two possibilities
	either cap(A, B) = cap(min cut)
	or cap(A, B) != cap(min cut)

If cap(A, B) = cap(min cut) then
	since cap(A, B) = |min vertex cover|
	cap(min cut) = |min vertex cover|
Else then
	cap(min cut) !> cap(A, B)
	So cap(min cut) < cap(A, B)
		Since cap(A, B) = |min vertex cover|
		then cap(min cut) < |min vertex cover|

Thus, given min vertex cover S
	we find s-t cut (A, B) such that
		cap(A, B) = |S| which implies that
		cap(A, B) = |min vertex cover|
	and
		cap(min cut) $\leq$ cap(A, B) which implies that 
		cap(min cut) $\leq$ |min vertex cover|

Combining the above, show algorithm finds min vertex cover from G
	Reset all variables
	Let bipartite graph G = (X, Y, E)
	Let flow network H be output of FlowNetwork(G)
	Let min s-t cut (A, B) be output of FFA(H)
	min cut (A, B) is necessary and sufficient for min vertex cover S
		since cap(min cut) $\leq$ |min vertex cover| $\leq$ cap(min cut)
	which implies that
		cap(min cut) = |min vertex cover|
	Therefore create min vertex cover S from min cut (A, B)
	Thus, the algorithm finds min vertex cover S from G
\end{lstlisting}



\end{document}