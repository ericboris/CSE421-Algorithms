\documentclass[11pt]{article}
\usepackage{amsmath, amsfonts, amsthm, amssymb}  % Some math symbols
\usepackage[utf8x]{inputenc}
\usepackage{fullpage}
\usepackage[x11names, rgb]{xcolor}
\usepackage{graphicx}
\usepackage{tikz}
\usepackage{etoolbox}
\usepackage{enumerate}
\usepackage{enumitem}
\usepackage{listings}
\usepackage{hyperref}
\usepackage{lipsum}
\usepackage{sectsty}
\usepackage{verbatim}
\usetikzlibrary{decorations,arrows,shapes}

%% Define the title contents
\title{}
\author{}
\date{}

%% Set the size of the section header
\sectionfont{\fontsize{11}{12}\selectfont}

%% Set the size and format of the subsection header
\subsectionfont{\fontsize{11}{12}\selectfont}
\renewcommand{\thesubsection}{\thesection (\alph{subsection})}

%% Set the size and format of the subsubsection header
\subsubsectionfont{\fontsize{9}{10}\selectfont}
\renewcommand{\thesubsubsection}{\roman{subsubsection}}

%% Define Real and Rational numbers symbol
\newcommand{\R}{\mathbb{R}}
\newcommand{\Q}{\mathbb{Q}}
\newcommand{\N}{\mathbb{N}}
\newcommand{\Z}{\mathbb{Z}}

%% Redefine rightarrow to imp
\def\imp{\rightarrow}

%% Redefine overline to ol
\def\ol{\overline}

%% Redefine leftrightarrow to lra
\def\lra{\leftrightarrow}

% Redefine setminus to sm
\def\sm{\setminus}

%% Define a nested environment using level for formal proof
\newenvironment{level}%
{\addtolength{\itemindent}{2em}}%
{\addtolength{\itemindent}{-2em}}



%% Set enumerate sub list to use numbers instead of letters
\setlist[enumerate]{label*=\arabic*.}

%% Define custom style
\lstdefinestyle{myCustomMatlabStyle}{
  language=Java,
  numbersep=10pt,
  tabsize=4,
  showspaces=false,
  showstringspaces=false
}

%% Define the default code language to Java
\lstset{basicstyle=\small, style=myCustomMatlabStyle}

%% Remove indentation at start of paragraph
\usepackage[parfill]{parskip}

%%--- Begin the Document ---%%

\begin{document}

\section*{P3}

\noindent\textcolor[RGB]{220,220,220}{\rule{\linewidth}{0.8pt}}

\subsection*{Prompt:} 

Let \lstinline{IndSetExists(G, k)} be a polynomial time algorithm that, given a graph G and an integer k, outputs \lstinline{true} if G has an independent set of size k and false otherwise. Now, given a graph G with n vertices and an integer $1 \leq k \leq n$ design a polynomial time algorithm that outputs an independent set of size k in G if it exists and outputs "Impossible" otherwise.

\subsection*{Algorithm:}


\begin{lstlisting}[basicstyle=\small, mathescape=true]
Input: A graph G with n vertices and an integer k
Output: An independent set of size k in G, if one exists
IndSet(G, k):
	If IndSetExists(G, k) = false then
		Return "Impossible"
	EndIf
	Let S be an empty set
	For each vertex v in G do 
		If k = 1 then
			S = S $\cup$ v
			Return S
		Else if IndSetExists((G - v), k) then
			G = G - {v}
		Else then
			G = G - {v} - N(v)
			S = S $\cup$ v
			k = k - 1
		EndIf
	EndFor
	Return S
\end{lstlisting}

\noindent\textcolor[RGB]{220,220,220}{\rule{\linewidth}{0.8pt}}
\linebreak

\subsection*{Claim:}

The algorithm terminates in polynomial time.

\subsection*{Proof:}

The algorithm loop iterates at most v times. On each loop iteration, \lstinline{IndSetExists}, which runs in polynomial time, is called at most once. Thus, the polynomial runtime of \lstinline{IndSetExists} dominates and the algorithm has a polynomial runtime.

\noindent\textcolor[RGB]{220,220,220}{\rule{\linewidth}{0.8pt}}

\subsection*{Claim:}

The algorithm outputs a set of vertices of size k, if one exists, or "Impossible" otherwise. 

\subsection*{Proof:}

Prove by induction. Let P($n$, $k$) be the claim that, given a graph $G$ with $n$ vertices and an integer $k$ such that $1 \leq k \leq n$, the algorithm outputs an independent set of $k$ vertices in $G$ if such a set exists, otherwise it outputs "Impossible". We concern ourselves here only with the non-trivial case where an independent set of size $k$ exists in $G$.  

Base Case: It's obvious that P($n$, $1$) holds for any $n \geq 1$.

Inductive Hypothesis: Assume P($i$, $j$) holds for $1 \leq i \leq n-1$ and $1 \leq j \leq k-1$. 

Inductive Step: Goal, show that P($n$, $k$) holds. Let $G$ be a graph with $n$ vertices and let $k$ be an integer such that $1 \leq k \leq n$. Since we know that $G$ contains an independent set of size $k$, our goal is only to show we can find such a set. Note the trivially obvious property, that if there is a subset $S \subseteq G$ and $|S| = k$ then $|G| \geq k$, where $|S|$ and $|G|$ denotes the size of the independent sets of $S$ and $G$, respectively. Let $S$ be such an independent subset $S \subseteq G$ where $|S| = k$. Let $v$ be an arbitrary vertex in $G$. Then, there are two cases, where $v \not\in S$ and where $v \in S$. We consider each in turn.  

Case 1: $v \not\in S$.
If $v \not\in S$, let $G' = G - \{v\}$. Since $G'$ has $n - 1$ vertices, $|G'| \leq k$. But, since $v \not\in S$, $S \subseteq G' \subseteq G$ and hence $|G'| \geq k$. Therefore $|G'| = k$. Thus since $|G'| = |S| = |G| = k$, P($n$, $k$) holds when $v \not\in S$.

Case 2: $v \in S$.
If $v \in S$, let $G' = G - \{v\} - N(v)$, where N($v$) denotes the vertex set of the neighbors incident to $v$. The inductive hypothesis holds for $G'$ so $|G'| = |S \sm v| = k - 1$. To show that $|S \cup v| = k$, consider adding $v$ back to $G$. Since $v$ and $N(v)$ were removed from $G$, $v$ was connected to $G$ by N($v$). By definition, no vertex $n \in N(v) \in S$. Therefore in $G$, $v$ was separated by at least two edges from any other vertex that appears in $S$. This implies that $|G' + v + N(v)| = |S \cup v| = |G| = k$. Thus, P($n$, $k$) holds when $v \in S$.

Because P($n$, $k$) holds in both cases, P($n$, $k$) holds generally. $\qed$
\end{document}