\documentclass[11pt]{article}
\usepackage{amsmath, amsfonts, amsthm, amssymb}  % Some math symbols
\usepackage[utf8x]{inputenc}
\usepackage{fullpage}
\usepackage[x11names, rgb]{xcolor}
\usepackage{graphicx}
\usepackage{tikz}
\usepackage{etoolbox}
\usepackage{enumerate}
\usepackage{enumitem}
\usepackage{listings}
\usepackage{hyperref}
\usepackage{lipsum}
\usepackage{sectsty}
\usepackage{verbatim}
\usetikzlibrary{decorations,arrows,shapes}

%% Define the title contents
\title{}
\author{}
\date{}

%% Set the size of the section header
\sectionfont{\fontsize{11}{12}\selectfont}

%% Set the size and format of the subsection header
\subsectionfont{\fontsize{11}{12}\selectfont}
\renewcommand{\thesubsection}{\thesection (\alph{subsection})}

%% Set the size and format of the subsubsection header
\subsubsectionfont{\fontsize{9}{10}\selectfont}
\renewcommand{\thesubsubsection}{\roman{subsubsection}}

%% Define Real and Rational numbers symbol
\newcommand{\R}{\mathbb{R}}
\newcommand{\Q}{\mathbb{Q}}
\newcommand{\N}{\mathbb{N}}
\newcommand{\Z}{\mathbb{Z}}

%% Redefine rightarrow to imp
\def\imp{\rightarrow}

%% Redefine overline to ol
\def\ol{\overline}

%% Redefine leftrightarrow to lra
\def\lra{\leftrightarrow}

% Redefine setminus to sm
\def\sm{\setminus}

%% Define a nested environment using level for formal proof
\newenvironment{level}%
{\addtolength{\itemindent}{2em}}%
{\addtolength{\itemindent}{-2em}}



%% Set enumerate sub list to use numbers instead of letters
\setlist[enumerate]{label*=\arabic*.}

%% Define custom style
\lstdefinestyle{myCustomMatlabStyle}{
  language=Java,
  numbersep=10pt,
  tabsize=4,
  showspaces=false,
  showstringspaces=false
}

%% Define the default code language to Java
\lstset{basicstyle=\small, style=myCustomMatlabStyle}

%%--- Begin the Document ---%%

\begin{document}

\section*{P2}

\noindent\textcolor[RGB]{220,220,220}{\rule{\linewidth}{0.8pt}}

\subsection*{Prompt:} 

Given a set of points $P = \{p_1, ..., p_n\} \in \R^d$, an integer k such that $1 \leq k \leq n$, and the optimum radius O($\Delta$). Design a polynomial time algorithm that finds at most k balls of radius O($\Delta$) centered at k points and covering all of the points. 

\subsection*{Algorithm:} 

\begin{lstlisting}[basicstyle=\small, mathescape=true]
FindSetA(P, O($\Delta$)):	
	Let set $S = \varnothing$
	Let radius $\Delta$ be 2 * O($\Delta$)
	While $P \neq \varnothing$ do
		Let point $p$ be arbitrary $p \in P$
		$S = S \cup p$
		For each point $u \in P$ do
			If distance$(p$, $u) \leq \Delta$ then
				$P = P \sm u$
			EndIf
		EndFor
	EndWhile
	Return S
\end{lstlisting}

\noindent\textcolor[RGB]{220,220,220}{\rule{\linewidth}{0.8pt}}
\linebreak

\subsection*{Claim:}

The algorithm terminates in polynomial time.

\subsection*{Proof:}

The algorithm has two nested loops. Both loops iterate over the set $P$ that is monotonically decreasing on each iteration. Thus, the algorithm terminates in O($n^2$). $\qed$

\noindent\textcolor[RGB]{220,220,220}{\rule{\linewidth}{0.8pt}}

\subsection*{Claim:}

Given a set of points $P$ such that $|P| = n$ and the optimum radius O($\Delta$) to enclose all P in k balls, the algorithm returns a set of points S of size $|S| = h \leq k$ centered at $h$ of the points and covering all of the points. 

\subsection*{Proof:}

The proof of the claim follows directly. Let O denote the optimum set of points such that all $p \in P$ are within O($\Delta$) radius from some point $o \in O$ and $|O| = k$. Let S denote the set of points returned by the algorithm at termination. Consider a point $s \in S$ such that $s \in O$. Then, a radius about s of O($\Delta$) is sufficient to enclose every point $p \in P$ that would also be enclosed by some point $o \in O$. Obviously, if every point $s \in S$ were a point $o \in O$ we would be done. Now consider the case where point $s \in S$ but $s \not\in O$. Let $\Delta$ =  2 * O($\Delta$). Consider the effect doubling O($\Delta$) has on the points enclosed within a radius $\Delta$ from s. Suppose $s$ is O($\Delta$) away from the nearest $o \in O$. Then, the furthest distance another point $q$ could be from $s$ and still be within O($\Delta$) of $o$ would be $\Delta$. Therefore, s with radius $\Delta$ encloses every point in P that is also enclosed by o with radius O($\Delta$). It follows then, that at most k points in S with radius $\Delta$ will be enough to enclose every point in P just as k points in O with radius O($\Delta$) were enough to enlose every point in P. $\qed$

\noindent\textcolor[RGB]{220,220,220}{\rule{\linewidth}{0.8pt}}

\subsection*{Prompt:}

Assume we no longer know O($\Delta$) but we know that O($\Delta$) is in the interval [1, R]. Design an n log(R) polynomial time algorithm to find k balls of radius O($\Delta$).

\subsection*{Algorithm:}

\begin{lstlisting}[basicstyle=\small, mathescape=true]
FindSetB(P, k):
	Let $\Delta$ = 1
	While $|S| > k$ do
		Let S = FindSetA(P, $\Delta$)
		$\Delta$ = $\Delta * 2$
	Return S
\end{lstlisting}

\subsection*{Claim:}

The algorithm terminates in n$^2$ log(R) polynomial time.

\subsection*{Proof:}

We know FindSetA terminates in $n^2$. Since $|S|$ decreases and $\Delta$ doubles on every iteration, we know the runtime will be in O($n^2$ log(R)). $\qed$

\subsection*{Claim:}

Given a set $P$ containing $n$ points and an integer k such that $1 \leq k \leq n$, the algorithm returns a set S of at most k points such that all the points in P are within at most c * O($\Delta$) radius, where c is some constant, from a point in S. 

\subsection*{Proof:}

Since the algorithm terminates, we know that $|S| \leq k$. We therefore show that $\Delta$ is within a constant c of O($\Delta$). We know that if $\Delta < O(\Delta)$ then $|S| > k$ and we know from part a that if $\Delta > 2 * O(\Delta)$ that $|S| \leq k$. Because we double $\Delta$ on each iteration, the largest $\Delta$ could be is $\Delta < 4 * O(\Delta)$. However, as shown, no iteration where $\Delta$ is larger than that will execute. Thus, $|S| \leq k$ and $\Delta$ is bounded by $O(\Delta) <= \Delta < 4 * O(\Delta)$. $\qed$
\end{document}