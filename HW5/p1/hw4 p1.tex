\documentclass[11pt]{article}
\usepackage{amsmath, amsfonts, amsthm, amssymb}  % Some math symbols
\usepackage[utf8x]{inputenc}
\usepackage{fullpage}
\usepackage[x11names, rgb]{xcolor}
\usepackage{graphicx}
\usepackage{tikz}
\usepackage{etoolbox}
\usepackage{enumerate}
\usepackage{enumitem}
\usepackage{listings}
\usepackage{hyperref}
\usepackage{lipsum}
\usepackage{sectsty}
\usepackage{verbatim}
\usetikzlibrary{decorations,arrows,shapes}

%% Define the title contents
\title{}
\author{}
\date{}

%% Set the size of the section header
\sectionfont{\fontsize{11}{12}\selectfont}

%% Set the size and format of the subsection header
\subsectionfont{\fontsize{11}{12}\selectfont}
\renewcommand{\thesubsection}{\thesection (\alph{subsection})}

%% Set the size and format of the subsubsection header
\subsubsectionfont{\fontsize{9}{10}\selectfont}
\renewcommand{\thesubsubsection}{\roman{subsubsection}}

%% Define Real and Rational numbers symbol
\newcommand{\R}{\mathbb{R}}
\newcommand{\Q}{\mathbb{Q}}
\newcommand{\N}{\mathbb{N}}
\newcommand{\Z}{\mathbb{Z}}

%% Redefine rightarrow to imp
\def\imp{\rightarrow}

%% Redefine overline to ol
\def\ol{\overline}

%% Redefine leftrightarrow to lra
\def\lra{\leftrightarrow}

% Redefine setminus to sm
\def\sm{\setminus}

%% Define a nested environment using level for formal proof
\newenvironment{level}%
{\addtolength{\itemindent}{2em}}%
{\addtolength{\itemindent}{-2em}}



%% Set enumerate sub list to use numbers instead of letters
\setlist[enumerate]{label*=\arabic*.}

%% Define custom style
\lstdefinestyle{myCustomMatlabStyle}{
  language=Java,
  numbersep=10pt,
  tabsize=4,
  showspaces=false,
  showstringspaces=false
}

%% Define the default code language to Java
\lstset{basicstyle=\small, style=myCustomMatlabStyle}

%%--- Begin the Document ---%%

\begin{document}

\section*{P1}

\noindent\textcolor[RGB]{220,220,220}{\rule{\linewidth}{0.8pt}}

\subsection*{Claim:} 

Any tournament of $2^n$ vertices contains an acyclic subtournament of at least $n + 1$ vertices. 

\subsection*{Proof:}

Prove by induction. Let P(n) be the claim that any tournament of $2^n$ vertices contains an acyclic subtournament of at least $n + 1$ vertices. 
\newline

Base case: P(0) holds because the subtournament of $2^{0} = 1$ vertices has $0 + 1 = 1$ vertices and is trivially acyclic.
\newline

Inductive Hypothesis: Assume that P($n-1$) holds for $n-1 > 0$. 
\newline

Inductive Step: Goal, show P($n$) holds. Let graph $G = (V, E)$ be a tournament with $2^n$ vertices. Let $v$ be an arbitrary vertex in $G$. Partition $G$ into two subtournaments $S_1$ and $S_2$ such that $S_1 = \{u \in V \sm v \hspace{0.2em}|\hspace{0.2em} (u, v) \in E\}$ and $S_2 = \{w \in V \sm v \hspace{0.2em}|\hspace{0.2em}  (v, w) \in E\}$. Because the in-degree of $v \geq 2^{n-1}$ or the out-degree of $v \geq 2^{n-1}$, $S_1$ or $S_2$ contains at least $2^{n-1}$ vertices. Let $S'$ be a tournament with $2^{n-1}$ vertices from $S_1$ or $S_2$. $S'$ has $2^{n-1}$ vertices, hence by the inductive hypothesis, $S'$ is an acyclic subtournament of $G$ with $n + 1 - 1$ vertices. Let tournament $S^* = S' \cup v$. $S^*$ is acyclic since $S'$ is a subset of $S_1$ or $S_2$ which were defined by their having only incoming or outgoing edges with $v$ and $S^*$ has $n + 1 - 1 + 1 = n + 1$ edges. Thus, $G$ has an acyclic subtournament $S^*$ with at least $n + 1$ vertices. $\qed$

\end{document}