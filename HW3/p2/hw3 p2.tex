\documentclass[11pt]{article}
\usepackage{amsmath, amsfonts, amsthm, amssymb}  % Some math symbols
\usepackage[utf8x]{inputenc}
\usepackage{fullpage}
\usepackage[x11names, rgb]{xcolor}
\usepackage{graphicx}
\usepackage{tikz}
\usepackage{etoolbox}
\usepackage{enumerate}
\usepackage{enumitem}
\usepackage{listings}
\usepackage{hyperref}
\usepackage{lipsum}
\usepackage{sectsty}
\usepackage{verbatim}
\usetikzlibrary{decorations,arrows,shapes}

%% Define the title contents
\title{}
\author{}
\date{}

%% Set the size of the section header
\sectionfont{\fontsize{11}{12}\selectfont}

%% Set the size and format of the subsection header
\subsectionfont{\fontsize{11}{12}\selectfont}
\renewcommand{\thesubsection}{\thesection (\alph{subsection})}

%% Set the size and format of the subsubsection header
\subsubsectionfont{\fontsize{9}{10}\selectfont}
\renewcommand{\thesubsubsection}{\roman{subsubsection}}

%% Define Real and Rational numbers symbol
\newcommand{\R}{\mathbb{R}}
\newcommand{\Q}{\mathbb{Q}}
\newcommand{\N}{\mathbb{N}}
\newcommand{\Z}{\mathbb{Z}}

%% Redefine rightarrow to imp
\def\imp{\rightarrow}

%% Redefine overline to ol
\def\ol{\overline}

%% Redefine leftrightarrow to lra
\def\lra{\leftrightarrow}

% Redefine setminus to sm
\def\sm{\setminus}

%% Define a nested environment using level for formal proof
\newenvironment{level}%
{\addtolength{\itemindent}{2em}}%
{\addtolength{\itemindent}{-2em}}



%% Set enumerate sub list to use numbers instead of letters
\setlist[enumerate]{label*=\arabic*.}

%% Define custom style
\lstdefinestyle{myCustomMatlabStyle}{
  language=Java,
  numbersep=10pt,
  tabsize=4,
  showspaces=false,
  showstringspaces=false
}

%% Define the default code language to Java
\lstset{basicstyle=\small, style=myCustomMatlabStyle}

%%--- Begin the Document ---%%

\begin{document}

\section*{P2}

\noindent\textcolor[RGB]{220,220,220}{\rule{\linewidth}{0.8pt}}

\subsection*{Prompt:} 
Given a sequence $\{d_1$, ..., $d_n\}$ of positive integers, construct a tree such that the degree of vertex $i$ is $d_i$ or output "impossible" if no such tree exists. 

\subsection*{Algorithm:} 
\begin{lstlisting}[basicstyle=\small]
// Construct a tree from given positive integer sequence K
DegreeTree(K):
	If Sum(K) does not equal 2(Length(K) - 1)
		Return "Impossible"
	Set S to SortDescending(K)
	Initialize tree T
	For each value s in S
		If tree T is empty then 
			Add root r to T
			Set r to degree s
		Else
			For node n in T starting from r
				If n has degree > 0
					Add leaf k to n
					Set k degree to s - 1
					Set n degree to n - 1
				EndIf
			EndFor
		EndIf
	EndFor
	Return T
\end{lstlisting}

\noindent\textcolor[RGB]{220,220,220}{\rule{\linewidth}{0.8pt}}
\linebreak

\subsection*{Claim:} 
The algorithm terminates in polynomial time. 

\subsection*{Proof:}
The algorithm has two loops. The outer loop iterates over each value in the given sequence exactly once for a runtime of O(n). The inner loop iterates over each node in the tree as it is being constructed at most once, for a runtime of O(n). Therefore in every case, the algorithm as a whole is bounded by O($n^2$). And, because each for loop is bounded by n, the algorithm is guaranteed to terminate. $\qed$

\noindent\textcolor[RGB]{220,220,220}{\rule{\linewidth}{0.8pt}}

\subsection*{Claim:}
For any sequence of positive integers $\{d_1$, ..., $d_n\}$ with $\sum_i d_i < 2n$ and with $d_i \geq 1$ for all $i$, there must be at least one $i$ with $d_i \leq 1$. 

\subsection*{Lemma:}
Assume for a contradiction that there is a sequence of positive integers $\{d_1$, ..., $d_n\}$ with $\sum_i d_i < 2n$ and with $d_i > 1$ for all $i$. Consider a sequence of positive integers of this form such that $d_i > 1$ for all $i$, let this sequence be $\{2_1$, $2_2$, ..., $2_n\}$. Then $\sum_i d_i = 2_1 + 2_2 + ... + 2_n = 2n < 2n$ which is a contradiction. $\qed$

\noindent\textcolor[RGB]{220,220,220}{\rule{\linewidth}{0.8pt}}

\subsection*{Claim:} 
Given a sequence of positive integers $\{d_1$, ..., $d_n\}$ the algorithm generates a tree with this degree sequence if and only if $\sum_i d_i = 2(n - 1)$, and for all $i$ we have $d_i \geq 1$ otherwise it outputs "Impossible". 

\subsection*{Proof:}
For a proof in the forward direction, if tree T was constructed from a sequence of positive integers $\{d_1$, ..., $d_n\}$ then T has $\sum_i d_i = 2(n - 1)$. For a contradiction assume that tree T was constructed from a sequence of positive integers $\{d_1$, ..., $d_n\}$ and that $\sum_i d_i \neq 2(n - 1)$. We know that T has n - 1 edges and know from lecture that the $\sum d(T) = 2m$. Therefore, T has n - 1 = m edges and degree $\sum d(T) = 2m = 2(n - 1)$ which is a contradiction. $\qed$ 

$\linebreak$
Prove the backwards direction by induction. Let P(n) be the claim that if a sequence of positive integers $\{d_1$, ..., $d_n\}$ has $\sum_i d_i = 2(n - 1)$ and $d_i \geq 1$ for all $i$ then a tree can be constructed from this sequence. 

Base case: We use n = 2 for a base case so that the condition $d_i \geq 1$ for all $i$ is met. P(2) holds because $\sum_i d_i = 1 = 2((2) - 1) = 2(n - 1)$ and $d_i \geq 1$ for all $i$.

Inductive Hypothesis: Assume P(n - 1) holds for all $n - 1 > 1$. 

Inductive Step: Goal show P(n). By the above lemma, for any sequence of positive integers $D = \{d_1$, ..., $d_n\}$ with $\sum_i d_i < 2n$ and with $d_i \geq 1$ for all $i$ there must be at least one $i$ with $d_i \leq 1$. Let this vertex with $d_i \leq 1$ be vertex v. Let $D' = D - v$. Note, that $d_i \geq 1$ for all $i$ in D' because removing v does not disconnect any vertices other than v. Then $\sum d(D') = 2(n - 1) - 2 = 2((n - 1) - 1)$. By the inductive hypothesis, we can create a tree T' from D'. Because T' is a tree with degree $2((n - 1) - 1)$, T = T' + v is also a tree with $\sum_i d_i = 2((n - 1) - 1) + 2 = 2(n - 1)$ and $d_i \geq 1$ for all $i$. $\qed$

$\linebreak$
Thus, because we've shown the biconditional holds, we've shown that the algorithm, which only executes if given a sequence of positive integers $\{d_1$, ..., $d_n\}$ such that $\sum_i d_i = 2(n - 1)$ and for all $i$ we have $d_i \geq 1$, generates a tree. $\qed$
\end{document}