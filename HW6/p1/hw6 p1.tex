\documentclass[11pt]{article}
\usepackage{amsmath, amsfonts, amsthm, amssymb}  % Some math symbols
\usepackage[utf8x]{inputenc}
\usepackage{fullpage}
\usepackage[x11names, rgb]{xcolor}
\usepackage{graphicx}
\usepackage{tikz}
\usepackage{etoolbox}
\usepackage{enumerate}
\usepackage{enumitem}
\usepackage{listings}
\usepackage{hyperref}
\usepackage{lipsum}
\usepackage{sectsty}
\usepackage{verbatim}
\usetikzlibrary{decorations,arrows,shapes}

%% Define the title contents
\title{}
\author{}
\date{}

%% Set the size of the section header
\sectionfont{\fontsize{11}{12}\selectfont}

%% Set the size and format of the subsection header
\subsectionfont{\fontsize{11}{12}\selectfont}
\renewcommand{\thesubsection}{\thesection (\alph{subsection})}

%% Set the size and format of the subsubsection header
\subsubsectionfont{\fontsize{9}{10}\selectfont}
\renewcommand{\thesubsubsection}{\roman{subsubsection}}

%% Define Real and Rational numbers symbol
\newcommand{\R}{\mathbb{R}}
\newcommand{\Q}{\mathbb{Q}}
\newcommand{\N}{\mathbb{N}}
\newcommand{\Z}{\mathbb{Z}}

%% Redefine rightarrow to imp
\def\imp{\rightarrow}

%% Redefine overline to ol
\def\ol{\overline}

%% Redefine leftrightarrow to lra
\def\lra{\leftrightarrow}

% Redefine setminus to sm
\def\sm{\setminus}

%% Define a nested environment using level for formal proof
\newenvironment{level}%
{\addtolength{\itemindent}{2em}}%
{\addtolength{\itemindent}{-2em}}



%% Set enumerate sub list to use numbers instead of letters
\setlist[enumerate]{label*=\arabic*.}

%% Define custom style
\lstdefinestyle{myCustomMatlabStyle}{
  language=Java,
  numbersep=10pt,
  tabsize=4,
  showspaces=false,
  showstringspaces=false
}

%% Define the default code language to Java
\lstset{basicstyle=\small, style=myCustomMatlabStyle}

%% Remove indentation at start of paragraph
\usepackage[parfill]{parskip}

%%--- Begin the Document ---%%

\begin{document}

\section*{P1}

\noindent\textcolor[RGB]{220,220,220}{\rule{\linewidth}{0.8pt}}

\subsection*{Claim:} 

Given the pictured graph with 2n vertices, the following dynamic program runs in polynomial time and outputs the number of independent sets of the graph. 

\subsection*{Algorithm:}

\begin{lstlisting}[basicstyle=\small, mathescape=true]
Input: The number of pairs in the pictured graph.
Output: The number of independent sets in the graph. 
IndSet(n):
	Let V be an empty list
	V[0] = 1
	V[1] = 3
	For 2 $\leq$ i $\leq$ n do
		V[i] = 2V[i - 1] + V[i - 2]
	EndFor
	Return V[n]
\end{lstlisting}

\noindent\textcolor[RGB]{220,220,220}{\rule{\linewidth}{0.8pt}}
\linebreak

\subsection*{Claim:} 

The algorithm runs in polynomial time.

\subsection*{Proof:}

That the algorithm terminates is obvious given that each element in the array is accessed once and only once. The number of steps to reach each point is cached as an array element as it's found, so during loop iteration, previously found step costs are accessed in constant time. Therefore the dominant runtime factor is the loop. Thus, the algorithm terminates in polynomial time. $\qed$

\noindent\textcolor[RGB]{220,220,220}{\rule{\linewidth}{0.8pt}}

\subsection*{Claim:} 

The algorithm correctly outputs the number of independent sets.

\subsection*{Proof:}

Let OPT(n, f) denote number of independent sets in the pictured graph, where n is the number of columns in the graph and f is a flag indicating whether the vertex in the independent set in the nth column is in the top row T, in the bottom row B, or in neither row N. Let OPT(n), with only one parameter, denote the sum of OPT(n, f) for each value of f. 

Case 1: If n=0 then there is only the empty set. 
\begin{align*}
\text{OPT(0)} &= \text{OPT(0, T) + OPT(0, B) + OPT(0, N)} \\
	&= \text{0 + 0 + 1} \\
	&= \text{1}
\end{align*}

Case 2: If n=1 then there are three possible independent sets.
\begin{align*}
\text{OPT(1)} &= \text{OPT(1, T) + OPT(1, B) + OPT(1, N)} \\
	&= \text{1 + 1 + 1} \\
	&= \text{3}
\end{align*}

Case 3: If n$>$1 then the total number of independent sets relies on previous solutions. 

First, consider the values of the two parameter OPT(n, f) at n.

If there is a vertex in the independent set in the top row of column n, then, to remain an independent set, any vertex in column n-1 must be in the bottom row or not be in the set at all. 
\begin{align*}
\text{OPT(n, T)} &= \text{OPT(n-1, B) + OPT(n-1 N)} \\
\end{align*}

Similar to OPT(n, T), if there is a vertex in the independent set in the bottom row of column n, then, to remain an independent set, any vertex in column n-1 must be in the top row or not be in the set at all.
\begin{align*}
\text{OPT(n, B)} &= \text{OPT(n-1, T) + OPT(n-1 N)} \\
\end{align*}
Note that OPT(n, B) and OPT(n, T) have the same number of solutions since one is a mirror reflection of the other. 

If there is no vertex in the independent set in column n, then OPT(n, N) is the sum of previous columns' solutions.
\begin{align*}
\text{OPT(n, N)} &= \text{OPT(n-1, T) + OPT(n-1, B) + OPT(n-1, N)} \\
	&= \text{OPT(n-1)}
\end{align*}

Now, combining these two parameter solutions we find the one parameter solution for OPT(n).
\begin{align*}
\text{OPT(n)} &= \text{OPT(n, T) + OPT(n, B) + OPT(n, N)} \\
	&= \text{OPT(n-1, B) + OPT(n-1 N) + OPT(n-1, T) + OPT(n-1 N) + OPT(n-1)} \\
	&= \text{OPT(n-1) + OPT(n-1) + OPT(n-1 N)} \\
	&= \text{2 * OPT(n-1) + OPT(n-2 N) + OPT(n-2, T) + OPT(n-2 N)} \\
	&= \text{2 * OPT(n-1) + OPT(n-2)} \\
\end{align*}

Thus, our recurrence relation for OPT(n):
\[
\text{OPT(n) =} 
\begin{cases} 
      \text{1} & \text{n=0} \\
      \text{3} & \text{n=1} \\
      \text{2 * OPT(n - 1) + OPT(n - 2)} & \text{n$>$1}
   \end{cases}
\]
$\qed$

\end{document}