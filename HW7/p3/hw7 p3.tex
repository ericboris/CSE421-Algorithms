\documentclass[11pt]{article}
\usepackage{amsmath, amsfonts, amsthm, amssymb}  % Some math symbols
\usepackage[utf8x]{inputenc}
\usepackage{fullpage}
\usepackage[x11names, rgb]{xcolor}
\usepackage{graphicx}
\usepackage{tikz}
\usepackage{etoolbox}
\usepackage{enumerate}
\usepackage{enumitem}
\usepackage{listings}
\usepackage{hyperref}
\usepackage{lipsum}
\usepackage{sectsty}
\usepackage{verbatim}
\usetikzlibrary{decorations,arrows,shapes}

%% Define the title contents
\title{}
\author{}
\date{}

%% Set the size of the section header
\sectionfont{\fontsize{11}{12}\selectfont}

%% Set the size and format of the subsection header
\subsectionfont{\fontsize{11}{12}\selectfont}
\renewcommand{\thesubsection}{\thesection (\alph{subsection})}

%% Set the size and format of the subsubsection header
\subsubsectionfont{\fontsize{9}{10}\selectfont}
\renewcommand{\thesubsubsection}{\roman{subsubsection}}

%% Define Real and Rational numbers symbol
\newcommand{\R}{\mathbb{R}}
\newcommand{\Q}{\mathbb{Q}}
\newcommand{\N}{\mathbb{N}}
\newcommand{\Z}{\mathbb{Z}}

%% Redefine rightarrow to imp
\def\imp{\rightarrow}

%% Redefine overline to ol
\def\ol{\overline}

%% Redefine leftrightarrow to lra
\def\lra{\leftrightarrow}

% Redefine setminus to sm
\def\sm{\setminus}

%% Define a nested environment using level for formal proof
\newenvironment{level}%
{\addtolength{\itemindent}{2em}}%
{\addtolength{\itemindent}{-2em}}



%% Set enumerate sub list to use numbers instead of letters
\setlist[enumerate]{label*=\arabic*.}

%% Define custom style
\lstdefinestyle{myCustomMatlabStyle}{
  language=Java,
  numbersep=10pt,
  tabsize=4,
  showspaces=false,
  showstringspaces=false
}

%% Define the default code language to Java
\lstset{basicstyle=\small, style=myCustomMatlabStyle}

%% Remove indentation at start of paragraph
\usepackage[parfill]{parskip}

%%--- Begin the Document ---%%

\begin{document}

\section*{P3}

\noindent\textcolor[RGB]{220,220,220}{\rule{\linewidth}{0.8pt}}

\subsection*{Prompt:} 

Given the described graph G, design an algorithm that runs in time polynomial in n and max$_e$c(e) and outputs "yes" if we can satisfy all demands and "no" otherwise. 

\subsection*{Algorithm:}
Let Abs(x) be the standard absolute value function that is given a real number x and  returns x if x $\geq$ 0 and -x if x $<$ 0.

Let FFA(G) be the Ford-Fulkerson algorithm which is given a network G = (V, E) with flow capacity c, a source node s, and a sink node t and returns the flow f from s to t of maximum value. 

\begin{lstlisting}[basicstyle=\small, mathescape=true]
Input: The described graph G
Output: "yes" if all demands can be satisfied and "no" otherwise
DemandSatisfied(G):
	Let H be a copy of G
	Add vertex s to H
	Add vertex t to H
	Set r(s) = 0
	Set r(t) = 0
	Let demand = 0
	
	For each vertex v in H do
		If r(v) > 0 then
			Add edge e from s to v with c(e) = r(v)
		Else if r(v) < 0 then
			Add edge e from v to t with c(e) = Abs(r(v))
			demand += Abs(r(v))
		EndIf
	EndFor
	
	Let maxFlow = FFA(H)
	If maxFlow >= demand then
		return "yes"
	Else then
		return "no"
	EndIf
\end{lstlisting}

\noindent\textcolor[RGB]{220,220,220}{\rule{\linewidth}{0.8pt}}
\linebreak

\subsection*{Claim:} 

The algorithm terminates in time polynomial in n and max$_e$c(e).

\subsection*{Proof:}

Since we assume that the capacities in G are integers, we know that Ford-Fulkerson is bounded by O(n * m * max$_e$c(e)) and hence has the same bound on H. The algorithm loop visits each vertex exactly once and so is bounded by the number of vertices in H. Therefore, the Ford-Fulkerson runtime dominates the loop runtime. Thus, the algorithm terminates in O(n * m * max$_e$c(e)).

\noindent\textcolor[RGB]{220,220,220}{\rule{\linewidth}{0.8pt}}

\subsection*{Claim:}

The algorithm outputs "yes" if and only if we can supply all the demands and "no" otherwise. 

\subsection*{Proof:}
First, consider the construction of the graph H within the algorithm. Let H be a graph that is a copy of the given arbitrary graph G with the following additions. Let H have two additional vertices s and t where r(s) = 0 and r(t) = 0. For each vertex v in H such that r(v) $>$ 0, let s have a directed edge e to v and let the capacity c(e) = r(v). For each vertex u in H such that r(u) $<$ 0, let u have a directed edge f to t and let the capacity c(f) = Abs(r(u)). 

Note the following facts about the construction of H:
\begin{enumerate}
\item The Sum\{c(e) for all edges e directed from s\} = Sum\{r(v) for all vertices v with r(v) $>$ 0 in H\} = the sum of all sources in G = the supply in G.
\item The Sum\{c(f) for all edges f directed to t\} = Sum\{Abs(r(u)) for all vertices u with r(u) $<$ 0 in H\} = the positive sum of all sinks in G = the demand in G. 
\end{enumerate}

Now, we prove the claim by considering the biconditional from both directions. 

If the algorithm is given a graph G with feasible supply/demand then the algorithm outputs "yes".
For a contradiction, assume that G has feasible supply/demand and the algorithm outputs "no"". 
By the construction of H, we know that since G has a feasible supply/demand that:
\begin{enumerate}
\item The sum of the capacity of edges leaving s is $\geq$ the sum of the capacity of the edges entering t and 
\item there is sufficient capacity between s and t to carry the current from s to t.
\end{enumerate}
Therefore, the maximum flow found by the Ford-Fulkerson algorithm on H will be $\geq$ the demand of G and the algorithm outputs "yes". Which is a contradiction. 

And from the other direction. 

If the algorithm outputs "yes" then it was given a graph G with feasible supply/demand. 
For a contrapositive, assume that G does not have feasible supply/demand and that the algorithm outputs "no". 
By the construction of H, we know that since G does not have feasible supply/demand that:
\begin{enumerate}
\item The sum of the capacity of edges leaving s is less than the sum of the capacity of the edges entering t or 
\item there is insufficient capacity between s and t to carry the current from s to t. 
\end{enumerate}
Therefore, the maximum flow found by the Ford-Fulkerson algorithm on H will be $<$ the demand of G and the algorithm outputs "no". 

Since we've shown both directions of the bidirectional, the claim holds. $\qed$
\end{document}