\documentclass[11pt]{article}
\usepackage{amsmath, amsfonts, amsthm, amssymb}  % Some math symbols
\usepackage[utf8x]{inputenc}
\usepackage{fullpage}
\usepackage[x11names, rgb]{xcolor}
\usepackage{graphicx}
\usepackage{tikz}
\usepackage{etoolbox}
\usepackage{enumerate}
\usepackage{enumitem}
\usepackage{listings}
\usepackage{hyperref}
\usepackage{lipsum}
\usepackage{sectsty}
\usepackage{verbatim}
\usetikzlibrary{decorations,arrows,shapes}

%% Define the title contents
\title{}
\author{}
\date{}

%% Set the size of the section header
\sectionfont{\fontsize{11}{12}\selectfont}

%% Set the size and format of the subsection header
\subsectionfont{\fontsize{11}{12}\selectfont}
\renewcommand{\thesubsection}{\thesection (\alph{subsection})}

%% Set the size and format of the subsubsection header
\subsubsectionfont{\fontsize{9}{10}\selectfont}
\renewcommand{\thesubsubsection}{\roman{subsubsection}}

%% Define Real and Rational numbers symbol
\newcommand{\R}{\mathbb{R}}
\newcommand{\Q}{\mathbb{Q}}
\newcommand{\N}{\mathbb{N}}
\newcommand{\Z}{\mathbb{Z}}

%% Redefine rightarrow to imp
\def\imp{\rightarrow}

%% Redefine overline to ol
\def\ol{\overline}

%% Redefine leftrightarrow to lra
\def\lra{\leftrightarrow}

% Redefine setminus to sm
\def\sm{\setminus}

%% Define a nested environment using level for formal proof
\newenvironment{level}%
{\addtolength{\itemindent}{2em}}%
{\addtolength{\itemindent}{-2em}}



%% Set enumerate sub list to use numbers instead of letters
\setlist[enumerate]{label*=\arabic*.}

%% Define custom style
\lstdefinestyle{myCustomMatlabStyle}{
  language=Java,
  numbersep=10pt,
  tabsize=4,
  showspaces=false,
  showstringspaces=false
}

%% Define the default code language to Java
\lstset{basicstyle=\small, style=myCustomMatlabStyle}

%% Remove indentation at start of paragraph
\usepackage[parfill]{parskip}

%%--- Begin the Document ---%%

\begin{document}

\section*{P1}

\noindent\textcolor[RGB]{220,220,220}{\rule{\linewidth}{0.8pt}}

\subsection*{Prompt:} 

Given a rectangular slab of size W x H and a set of rectangular plates of sizes W$_1$ x H$_1$, W$_2$ x H$_2$, ... W$_n$ x H$_n$, design an algorithm that runs in time polynomial in n, W, and H and that outputs the minimum possible waste area. 

\subsection*{Algorithm:}

\begin{lstlisting}[basicstyle=\small, mathescape=true]
Input: Slab dimensions W and H and a set of plates P. 
Output: The minimum possible waste area. 

Let t be an (W + 1) x (H + 1) table
Let every entry in t = -1

maxCut(W, H, P):
	If t[W][H] != -1 then
		Return t[W][H]
	EndIf
	
	Let C be a copy of P
	For each plate p in P do
		If p.w = W and p.h = H then
			Set t[W][H] = W * H
			Return t[W][H]
		Else if W $<$ p.w or H $<$ p.h then
			Remove p from C
		EndIf
	EndFor
	
	Let vertCutMax = 0
	Let horizCutMax = 0
	
	For each plate p in C do
		If p.w != W then
			Let vertCut = maxCut(W - p.w, H, C) + maxCut(p.w, H, C)
			Set vertCutMax = max{vertCut, vertCutMax}
		EndIf
	
		If p.h != H then
			Let horizCut = maxCut(W, H - p.h, C) + maxCut(W, p.h, C)
			Set horizCutMax = max{horizCut, horizCutMax}
		EndIf
	EndFor
	
	Let maxCut = max{vertCutMax, horizCutMax}
	t[W][H] = maxCut
	return t[W][H]
	
Let maxCut = maxCut(W, H, P)
Output (W * H) - maxCut
\end{lstlisting}

\noindent\textcolor[RGB]{220,220,220}{\rule{\linewidth}{0.8pt}}
\linebreak

\subsection*{Claim:} 

The algorithm terminates in time polynomial in n, W, and H. 

\subsection*{Proof:}

The algorithm iterates over each plate p in P at most 2n times. In the first of these loops, the work performed is constant in time, making the dominant runtime factor the n loop iterations. In the second loop, the work is divided into 4 subproblems: 2 subproblems of size W-p.w and p.w and 2 sub problems of size H-p.h and p.h, respectively. In each case, when a solution to a subproblem is found it is stored in the (W+1) x (H+1) table making future checks of that same sized subproblem accessible in constant time. Therefore, the runtime of the problem is bounded by W * H * 2n since there are W * H entries that must be checked and for each entry as many as 2n checks against the n plates in P. Since 2 is a constant factor, we ignore that and can say that the runtime of the problem is bounded by W * H * n. Since each subproblem is monotonically smaller in size than the calling problem, since each loop iterates at most n times, and since the problems are bounded on the low end (a plate p is removed from P if it's larger than W x H), the algorithm terminates. Thus, that algorithm terminates in time polynomial in n, W, and H. 

\noindent\textcolor[RGB]{220,220,220}{\rule{\linewidth}{0.8pt}}

\subsection*{Claim:} 

The algorithm outputs the minimum possible waste area. 

\subsection*{Proof:}

Let OPT(W, H) denote the maximum usable area cut from a slab with dimensions W x H into plates from a set P = \{W$_1$ x H$_1$, W$_2$ x H$_2$, ... W$_n$ x H$_n$\}.

Let v(W, H) be a function that takes the slab dimensions W x H and returns the area of that slab if there is a plate p of size W x H in P and returns 0 otherwise.
\[
\text{v(W, H) =} 
\begin{cases} 
      \text{W * H} & \text{if p.w = W and p.h = H for any p in P} \\
      \text{0} & \text{otherwise} \\
\end{cases}
\]

Case 1: If a slab with dimensions W x H is not cut, then the maximum usable area of that slab is W * H if there there is a plate p in P such that p.w = W and p.h = H. Otherwise, if no such plate p exists, then the maximum usable area of a slab with dimensions W x H is 0. This is given by the defined v(W, H) function.
\begin{align*}
\text{OPT(W, H)} &= \text{v(W, H)} \\
\end{align*}

Case 2: If a slab with dimensions W x H is cut vertically, then the cut that yields the maximum usable area for the slab with dimesions W x H will be the cut yielding two slabs with dimensions (W - p.w, H) and (p.w, H) for some plate p in P such that the sum of the maximum usable area of the two cut slabs is the maximum over all plates p in P.
\begin{align*}
\text{OPT(W, H)} &= \text{max\{OPT(W - p.w, H) + OPT(p.w, H)\} for all p in P} \\
\end{align*}

Case 3: If a slab with dimensions W x H is cut horizontally, then the cut that yields the maximum usable area for the slab with dimesions W x H will be the cut yielding two slabs with dimensions (W, H - p.h) and (W, p.h) for some plate p in P such that the sum of the maximum usable area of the two cut slabs is the maximum over all plates p in P.
\begin{align*}
\text{OPT(W, H)} &= \text{max\{OPT(W, H - p.h) + OPT(W, p.h)\} for all p in P} \\
\end{align*}

Thus, the recurrence relation that maximizes the usable area of the slab:
\[
\text{OPT(W, H) = max} 
\begin{cases} 
      \text{v(W, H)} & \text{If not cut}\\
	  \text{max\{OPT(W - p.w, H) + OPT(p.w, H)\} for all p in P} & \text{If cut vertically}\\
	  \text{max\{OPT(W, H - p.h) + OPT(W, p.h)\} for all p in P} & \text{If cut horizontally}\\
\end{cases}
\]

To find the minimal wasted area, simply subtract the maximum usable area given by OPT(W, H) from the total area of the slab, W * H. $\qed$
\end{document}